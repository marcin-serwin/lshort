% The Not So Short Introduction to LaTeX
%
% Copyright (C) 1995--2022 Tobias Oetiker, Marcin Serwin, Hubert Partl,
% Irene Hyna, Elisabeth Schlegl and Contributors.
%
% This document is free software: you can redistribute it and/or modify it
% under the terms of the GNU General Public License as published by the Free
% Software Foundation, either version 3 of the License, or (at your option) any
% later version.
%
% This document is distributed in the hope that it will be useful, but WITHOUT
% ANY WARRANTY; without even the implied warranty of MERCHANTABILITY or FITNESS
% FOR A PARTICULAR PURPOSE.  See the GNU General Public License for more
% details.
%
% You should have received a copy of the GNU General Public License along with
% this document.  If not, see <https://www.gnu.org/licenses/>.

% !TEX root = ./lshort.tex

\chapter{\LaTeX{} Basics}\label{chap:basics}
\begin{intro}
  The first part of this chapter presents a short
  overview of the philosophy and history of \LaTeX. The second part
  focuses on the basic structures of a \LaTeX{} document.
  After reading this chapter, you should have a rough knowledge
  of how \LaTeX{} works, which you will need to understand the rest
  of this book.
\end{intro}

\section{A Bit of History}
\subsection{\TeX}

\TeX{} is a computer program created by Donald E. Knuth\index{Knuth, Donald
  E.}~\cite{texbook}. The original program was aimed at typesetting text and
mathematical formulae. Knuth started writing the \TeX{} typesetting engine in
1977 to explore the potential of digital printing equipment. These new devices
were beginning to infiltrate the publishing industry at that time.
His goal was to reverse the trend of deteriorating typographical
quality that he saw affecting his own books and articles. The first stable
version of \TeX{} was released in 1982. Version 3.0 was released in 1989 to
better support 8-bit characters and multiple languages. Knuth considered the
\TeX-design to be complete with the release of Version 3. \TeX{} is renowned
for being extremely stable, for running on many kinds of computers,
and for being virtually bug free. The version number of \TeX{} is converging
to the \(\pi\) constant and is now at \(3.141592653\).

\TeX{} is pronounced \enquote{Tech}, with a \enquote{ch} as in the German word
\enquote{Ach}\footnote{In German there are actually two pronunciations for
  \enquote{ch} and one might assume that the soft \enquote{ch} sound from
  \enquote{Pech} would be a more appropriate. When asked about this by one of
  the German Wikipedia contributors, Knuth wrote:
  \textquote[\cite{germanwikiknuth}]{I do not get angry when people pronounce
    \TeX{} in their favorite way\ldots{}and in Germany many use a soft ch
    because the {\fontspec{cmunrm.otf}χ} follows the vowel e, not the harder ch
    that follows the vowel a. In Russia, \enquote*{tex} is a very common word,
    pronounced \enquote*{tyekh}. But I believe the most proper pronunciation is
    heard in Greece, where you have the harsher ch of ach and Loch.}} or in the
Scottish \enquote{Loch}' The \enquote{ch} originates from the Greek alphabet
where X is the letter \enquote{ch} or \enquote{chi}. \TeX{} is also the first
syllable of the Greek word technique. In an ASCII environment, \TeX{} becomes
\texttt{TeX}.

\subsection{Other \TeX{} engines}

While the original \TeX{} engine is still fully functional, and can be used to
typeset documents, some of the technical solutions it uses are now dated. Over
the years, the \TeX{} engine has been extended, introducing many new features.
Due to the way \TeX{} is licensed, anyone is free to produce enhanced versions
of \TeX{}, but they must not call the program \TeX{} anymore. The first widely
popular enhanced version of \TeX{} was \hologo{eTeX}~\cite{etex}.

The original program produced \eei{.dvi} files which were meant only to be sent
to a printer. With the proliferation of high resolution displays, it became more
common to read documents directly on-screen without printing them. This
prompted creation of another extension, called \hologo{pdfTeX}, which could
produce standard PDF files. Yet another problem was the original font format,
which was not compatible with modern font formats. This was in turn solved in
\hologo{XeTeX}.

Today, four \TeX{} engines are actively maintained: the original \TeX{},
\hologo{pdfTeX}, \hologo{XeTeX} and \hologo{LuaTeX}. This book recommends using
either \hologo{XeTeX} or \hologo{LuaTeX}. The examples presented should produce
the same results on both engines (except where otherwise noted). The basic
examples will work with \hologo{pdfTeX}\footnote{The original \TeX{} will not
  work at all, because \LaTeX{} requires an \hologo{eTeX}-enabled engine since
  2017~\cite{etex-kernel}.} too, but we suggest to switch to \hologo{XeTeX} or
\hologo{LuaTeX} from the start, to avoid complications down the road as you
explore more advanced concepts.

\subsection{\LaTeX{}}

\LaTeX{} is a set of macros\footnote{Macros are short names for long lists of
  instructions, which are created to avoid retyping the instructions each time
  they are needed.} for the \TeX{} engine. \LaTeX{} was originally developed by
Leslie Lamport\index{Lamport, Leslie} for his own use. After some
consideration, he decided to make them more general so that others could use
them for their own projects. Thus, in 1985, the first version of \LaTeX{} ---
named \LaTeX{} 2.09 --- was released~\cite{manual}.

The original \LaTeX{} became quite popular and promoted the creation of many
extension packages. Unfortunately, some of the more popular extensions were not
compatible with each other. \LaTeXe{} managed to unify many of the extensions,
and also provided an extension packaging system, dealing with third party
extensions in a standardised way.

The same year \LaTeXe{} was released, the \LaTeX3 project was started. Its aim
was to create improved standards for writing \LaTeX{} documents, fixing some of
the mistakes that were made when defining the initial \LaTeX{} macros. While at
the beginning, it was planned to release \LaTeX3 as a standalone system that was
not backward compatible with \LaTeXe{}, in the end, the consensus was that
abandoning the huge collection of third party packages written for \LaTeXe{},
would be a mistake. Thus, the development team decided that
\LaTeX3 would be slowly backported into \LaTeXe{} format, while avoiding
breaking changes as much as reasonably possible~\cite{quovadis}.

\LaTeX{} is pronounced \enquote{Lay-tech} or \enquote{Lah-tech.} If you refer
to \LaTeX{} in an ASCII environment, you type \texttt{LaTeX}. \LaTeXe{} is
pronounced \enquote{Lay-tech two e} and typed \texttt{LaTeX2e}. \LaTeX3 is
pronounced \enquote{Lay-tech three} and typed \texttt{LaTeX3}.

\section{Basics}

\subsection{Author, Book Designer, and Typesetter}

To publish something, authors give their typed manuscript to a
publishing company. One of their book designers then
decides the layout of the document (column width, fonts, space before
and after headings,~\ldots). The book designer writes his instructions
into the manuscript and then gives it to a typesetter, who typesets the
book according to these instructions.

A human book designer tries to find out what the author had in mind
while writing the manuscript. He decides on chapter headings,
citations, examples, formulae, etc., based on his professional
knowledge and from the contents of the manuscript.

In a \LaTeX{} environment, \LaTeX{} takes the role of the book designer and
uses \TeX{} as its typesetter. But \LaTeX{} is \enquote{only} a program and
therefore needs more guidance. The author has to provide additional information
to describe the logical structure of his work. This information is written into
the text as \enquote{\LaTeX{} commands.}

This is quite different from the \wi{WYSIWYG}\footnote{What You See Is
  What You Get.} approach that most modern word processors, such as
\emph{MS Word} or \emph{LibreOffice}, take. With these
applications, authors specify the document layout interactively while
typing text into the computer. They can see on the
screen how the final work will look when it is printed.

When using \LaTeX{}, it is not normally possible to see the final output
while typing the text, but the final output can be previewed on the
screen after processing the file with \LaTeX. Corrections can then be
made before actually sending the document to the printer.

\subsection{Layout Design}\label{sec:layout_design}

Typographical design is a craft. Unskilled authors often commit
serious formatting errors by assuming that book design is mostly a
question of aesthetics---\enquote{If a document looks good artistically,
  it is well-designed.} But as a document has to be read and not hung
up in a picture gallery, the readability and comprehensibility is
much more important than the beautiful look of it.
Examples:
\begin{itemize}
  \item The font size and the numbering of headings have to be chosen to make
        the structure of chapters and sections clear to the reader.
  \item The line length has to be short enough to not strain
        the eyes of the reader, while long enough to fill the page
        beautifully.
\end{itemize}

With \wi{WYSIWYG} systems, authors tend to generate aesthetically
pleasing documents with very little, or inconsistent, structure.
\LaTeX{} prevents such problems by forcing the author to
declare the \emph{logical} structure of his document. \LaTeX{} then
chooses the most suitable layout.

\subsection{Advantages and Disadvantages}

When people from the \wi{WYSIWYG} world meet people who use \LaTeX{},
they often discuss \enquote{the \wi{advantages of \LaTeX{}} over a normal
  word processor}, or the opposite.  The best thing to do when such
a discussion starts, is to keep a low profile, since such discussions
like to get out of hand. But sometimes there is no escaping \ldots

\medskip\noindent So here is some ammunition. The main advantages
of \LaTeX{} over normal word processors are the following:

\begin{itemize}

  \item Professionally crafted layouts are available, which make a
        document really look as if \enquote{printed}.
  \item The typesetting of mathematical formulae is supported out of the box.
  \item Users only need to learn a few easy-to-understand commands
        that specify the logical structure of a document. They almost never
        need to tinker with the actual layout of the document.
  \item Even complex structures, such as footnotes, references, table of
        contents, and bibliographies, can be generated easily.
  \item Free add-on packages exist for many typographical tasks not directly
        supported by basic \LaTeX. For example, packages are available to
        include \PSi{} graphics or to typeset bibliographies conforming to
        exact standards. Many of these add-on packages are described in
        \companion.
  \item \LaTeX{} encourages authors to write well-structured texts,
        because this is how \LaTeX{} works---by specifying structure.
  \item \TeX, the formatting engine of \LaTeX, is highly portable and free.
        Therefore, the system runs on almost any hardware platform
        available.

        %
        % Add examples ...
        %
\end{itemize}

\medskip

\noindent\LaTeX{} also has some disadvantages, and I guess it's a bit
difficult for me to find any sensible ones, though I am sure other people
can tell you hundreds \smiley.

\begin{itemize}
  \item \LaTeX{} does not work well for people who have sold their
        souls \ldots
  \item Although some parameters can be adjusted within a predefined
        document layout, the design of a whole new layout is difficult and
        takes a lot of time.
  \item It is very hard to write unstructured and disorganised documents.
  \item Your hamster might, despite some encouraging first steps, never be
        able to fully grasp the concept of Logical Markup.
\end{itemize}

\section{\LaTeX{} Input Files}

The input for \LaTeX{} is a plain text file. On \Unix{} systems text files are
pretty common. On Windows, one could use Notepad to create a text file. It
contains the text of the document, as well as the commands that tell \LaTeX{}
how to typeset the text. For beginners, it is recommended to use a \LaTeX{}
IDE.\footnote{Some examples of these are listed in
  \autoref{installinglatex}.}

\subsection{Spaces}\label{sec:spaces}

\enquote{Whitespace} characters, such as blank or tab, are treated uniformly as
\enquote{\wi{space}} by \LaTeX{}. Several consecutive \wi{whitespace}
characters are treated as \emph{one} \enquote{space}. Whitespace at the start
of a line is generally ignored, and a single line break has the same effect as
\enquote{whitespace}\index{whitespace!at the start of a line}.

An empty line between two lines of text defines the end of a paragraph. Multiple
empty lines are treated the same as \emph{one} empty line. The text below is an
example. On the left-hand side is the text from the input file, and on the
right-hand side is the formatted output.

\begin{example}
It does not matter whether you
enter one or many     spaces
after a word.

An empty line starts a new
paragraph.
\end{example}

\subsection{Comments}\index{comments}\label{sec:comments}

When \LaTeX{} encounters a \ai{\%} character while processing an input file,
it ignores the rest of the present line, the line break, and all
whitespace at the beginning of the next line.

This can be used to write notes into the input file, which will not show up
in the printed version.

\begin{example}
This is an % stupid
% Better: instructive <----
example: Supercal%
              ifragilist%
    icexpialidocious
\end{example}

The \ai{\%} character can also be used to split long input lines where no
whitespace or line breaks are allowed.

\subsection{Special Characters}

The following symbols are \wi{reserved characters} that have a special meaning
under \LaTeX{}. If you enter them directly in your text, they will normally not
print, but rather coerce \LaTeX{} to do things you did not intend.
\begin{code}
\verb.#  $  %  ^  &  _  {  }  ~  \ . %$
\end{code}

As you will see, these characters can be used in your documents all
the same by using a prefix backslash:

\begin{example}
\# \$ \% \^{} \& \_ \{ \} \~{}
\textbackslash{}
\end{example}

Many more other symbols and many more can be printed with special commands
in mathematical formulae or as accents. The backslash character
\textbackslash{} can \emph{not} be entered by adding another backslash
in front of it (\csi{\bs}); this sequence is used for
line breaking. Use the \csi{textbackslash} command instead.

\subsection{\LaTeX{} Commands}

\LaTeX{} \wi{commands} are case-sensitive, and take one of the following
two formats:

\begin{itemize}
  \item They start with a \wi{backslash} \verb|\| and then have a name
        consisting of letters only. Command names are terminated by a space, a
        number or any other \enquote*{non-letter}, for example:
        \begin{chktexignore}
          \mintinline{latex}|\emph|, \mintinline{latex}|\begin|,
          \mintinline{latex}|\LaTeX|.
        \end{chktexignore}
  \item They consist of a backslash and exactly one non-letter, for example:
        \begin{chktexignore}
          \mintinline{latex}|\\|, \mintinline{latex}|\{|, \mintinline{latex}|\"|.
        \end{chktexignore}
\end{itemize}
Many commands also exist in a `starred variant' where a star is appended to the
command name.

\subsection{Groups}

Many commands act on parameters. Parameters are the first things the command
encounters after its name ends in the source file. If the command name consists
of letters, then it ignores following spaces. For example, if command
\enquote*{foo} accepts two arguments, then
\begin{code}
  \mintinline{latex}+\foo bar+
\end{code}
will be read as command \enquote*{foo} with first argument \enquote*{b} and
second argument \enquote*{a} followed by the letter \enquote*{r}. In order to
pass more than one letter as a parameter, groups are used.

Groups are delimited by \ai{\{} and \ai{\}}. They tell \LaTeX{} to treat the
content between them as a single unit. For example,
\begin{code}
  \mintinline{latex}+\foo{bar}{baz}qux+
\end{code}
will be interpreted as command \enquote*{foo} with first argument
\enquote*{bar} and second argument \enquote*{baz} followed by text
\enquote*{qux}. Always using groups when passing parameters makes the
source code easier to read.

Commands that do not take any parameters still ignore any spaces after them.
The easiest way to stop this behaviour, is to follow them by an empty group.

\begin{chktexignore}
  \begin{example}[examplewidth=0.45\linewidth]
New \TeX users may miss
the whitespace
after a command. % renders wrong
Experienced \TeX{} users are
\TeX nicians, and know how to use
whitespace. % renders correct
\end{example}
\end{chktexignore}

\subsection{Optional parameters}

Many \LaTeX{} commands also accept optional parameters. The optional parameters
are normally enclosed
within square brackets \verb|[ ]|, and usually come right after the command
name. The following notation will often be used within this book to denote a
command with one optional parameter and one required parameter:
\begin{code}
  \cs{command}[optional parameter: o, parameter: m]
\end{code}

\section{Input File Structure}\label{sec:structure}
When \LaTeX{} processes an input file, it expects it to follow a
certain \wi{structure}. Input files must start with the
command
\begin{code}
  \csi{documentclass}[class: m]
\end{code}
The \carg{class} argument specifies what sort of document you intend to write.
Available classes are listed in \autoref{documentclasses}. Usually the
\cli{article} class is sufficient.

\begin{table}
  \caption{Document Classes.}\label{documentclasses}
  \begin{tabular}{@{}lp{8cm}@{}}
    \toprule
    Class         & Description                                       \\
    \midrule
    \cli{article} & for articles in
    scientific journals, short reports, and any other short document. \\
    \cli{proc}    & a class for
    proceedings based on the article class.                           \\
    \cli{report}  & for longer reports.
    containing several chapters, small books, PhD theses, \ldots      \\
    \cli{book}    & for real books.                                   \\
    \cli{beamer}  & for presentations.                                \\
    \cli{letter}  & for letters.                                      \\
    \bottomrule
  \end{tabular}
\end{table}

Now begins an area of the input file that is called the \emph{\wi{preamble}}.
Inside it, you add commands to influence the style of the whole document, or
load \wi{package}s that add new features to the \LaTeX{} system. To load such a
package, you use the command
\begin{code}
  \csi{usepackage}[package: m]
\end{code}

When the preamble is finished, you start the \emph{body} of the text with the
command
\begin{code}
  \ltx|\begin{document}|
\end{code}

Inside the body, you enter the text mixed with some useful \LaTeX{} commands.
Mark the end of the document with the
\begin{code}
  \ltx|\end{document}|
\end{code}
command, which tells \LaTeX{} to call it a day. Anything that
follows this command will be ignored by \LaTeX.

\autoref{mini} shows the contents of a minimal \LaTeX{} file. A
slightly more complicated \wi{input file} is given in
\autoref{document}.

\begin{listing}
  \begin{lined}{10cm}
    \begin{example}[standalone, template=empty, noextend]
\documentclass{article}

\usepackage[paperheight=\height,paperwidth=\width,margin=1cm,includefoot]{geometry} %!hide
\begin{document}
Small is beautiful.
\end{document}
\end{example}
  \end{lined}
  \caption{A Minimal \LaTeX{} File.}\label{mini}
\end{listing}

\begin{listing}
  \begin{lined}{\textwidth}
    \begin{example}[standalone, template=empty, noextend]
\documentclass[a4paper,11pt]{article}

\usepackage[paperheight=2\height,paperwidth=2\width,margin=0.8cm,includefoot]{geometry} %!hide
\author{H.~Partl}
\title{Minimalism}

\begin{document}
\maketitle
\tableofcontents

\section{Some Interesting Words}
Well, and here begins my lovely
article.

\section{Goodbye World}
\ldots{} and here it ends.

\end{document}
\end{example}
  \end{lined}
  \caption[Example of a realistic journal article.]{Example of a realistic
    journal article. Note that all the commands you see in this example will be
    explained later.}\label{document}
\end{listing}

\section{A Typical Command Line Session}

I bet you must be dying to try out the neat small \LaTeX{} input file shown on
\autopageref{mini}. Here is some help: \LaTeX{} itself comes without a GUI or
fancy buttons to press. It is just a program that crunches away at your input
file. Some \LaTeX{} installations feature a graphical front-end where there is
a \LaTeX{} button to start compiling your input file. On other systems, there
might be some typing involved, so here is how to coax \LaTeX{} into compiling
your input file on a text based system. Please note: this description assumes
that a working \LaTeX{} installation already sits on your computer. If this is
not the case, you may want to look at \autoref{installinglatex} on
\autopageref{installinglatex} first.

\begin{enumerate}
  \item Edit/Create your \LaTeX{} input file. This file must be plain text.  On
        \Unix{} systems, most of the editors will create just that. On Windows,
        you might want to make sure that you save the file in \emph{Plain Text}
        format. When picking a name for your file, make sure it bears the
        extension \eei{.tex}.

  %$ Should this be '\texttt{cmd} window' (or some such formatting)?
  \item Open a shell or cmd window, \texttt{cd} to the directory where your
        input file is located and run \LaTeX{} on your input file using either
        \begin{lscommand}
          \verb+xelatex foo.tex+
        \end{lscommand}
        or
        \begin{lscommand}
          \verb+lualatex foo.tex+
        \end{lscommand}
\end{enumerate}

If successful, you will end up with a \eei{.pdf} file. It may be necessary to
run \LaTeX{} several times to get the table of contents and all internal
references right. When your input file has a bug, \LaTeX{} will tell you about
it and stop processing your input file. Type \texttt{ctrl-D} to get back to the
command line.

\section{Logical Structure of Your Document}\label{sec:logical_structure}

In \autoref{sec:layout_design}, we mentioned that one of the
differences between \LaTeX{} and WYSIWYG editors is that in \LaTeX{} you write
the document by specifying its logical structure. This section will explore
this idea in more detail by presenting a problem and demonstrating how it can be
solved with logical markup. If you are familiar with the idea (for example, you
have worked with HTML and CSS), you can safely skip this section.

%$ Never-ending is standard English, but Neverending Story has crept in from
%$ unendliche, which I guess is what you're going for here.
\subsection{A Neverending Story of Problems with WYSIWYG Editors}

Let us imagine that you are writing a novel in your favourite WYSIWYG editor. In
this book, there are two parallel storylines happening in different dimensions.
In order to communicate to the reader which of the dimensions is currently
described, you have decided to use different colours for them. Thus, your book
may look like this:
\begin{quotation}
  {\color[HTML]{B71C1C}

    \enquote{No, thou canst leave me!} shouted Peredur to the Launcelot.

    \enquote{I have to} he replied. \enquote{Destiny calls upon me. But we shall
      meet again. I promise.} }

  {\color[HTML]{2E7D32}

    Shao felt inexplicable sadness, as if they had just lost something or
    someone important to them.

    \enquote{Are you okay?} asked Ashby while eating her slurry. \enquote{You
      don't look well.}

    \enquote{I'm fine, just tired.}
  }
\end{quotation}

After you finish your first draft, you decided to email it over to your friend
so they can share their opinion about it. However, it turns out that your
friend's printer can only print black and white, so they cannot print the file
you sent them.

After some consideration, you decided to simply use cursive font for one
dimension, while keeping upright font for the other. After some time of manually
changing each paragraph of your book to match the correct font, you remembered
that you have already used cursive font for emphasis in some cases, such as
\enquote{{\color[HTML]{2E7D32} Are you gonna eat \emph{that}?}}. Continuing
with your current approach you, would also have to check each paragraph for
emphasis and change it to something else before changing the font to cursive.

The above problems with changing the style of paragraphs are caused by the fact
that your WYSIWYG editor doesn't know that all green paragraphs somehow
represent common concept (events in one dimension). It just remembers that you
want these words in green, those in red and that one in cursive. Thus, changing
it to a different style means going over everything again, and manually changing
the style of each and every paragraph and word.

Wouldn't it be nice if you could somehow communicate to your editor
\enquote{these paragraphs are happening in dimension A}, and then simply decide
that all such paragraphs are green or use a different font? This is exactly
what logical markup does.

\subsection{Your First Text Command}

To see how this example would play out differently in \LaTeX{}, let's introduce
our first text command:
\begin{lscommand}
  \csi{emph}[text: m]
\end{lscommand}
It stands for \enquote{emphasise} and does just that: it emphasises the
\carg{text} it received as a parameter.
\begin{example}
Are you gonna eat \emph{that}?
\end{example}
If you write the code on the left in the body of your document (you can use
\autoref{mini} as a template), you will get the output on the right in the
PDF file you produce.

As you can see, \LaTeX{} typesets the emphasised text in cursive font by
default. It is important to understand that the \csi{emph} command
\emph{does not} mean \enquote{write this text in cursive}. It is much smarter
than that. To illustrate this, let's see how \csi{emph} behaves when used inside
text that is already in cursive
\begin{example}
  \itshape%!hide
Are you gonna eat \emph{that}?
\end{example}
As you can see, in this case it changed the font back to upright.

Remember that you should only use the \csi{emph} command to emphasise text and
nothing else, even if the resulting output would be the same (for example
cursive font). If you stick to this rule, then if you later decide to use a
different form of emphasis, you can simply change the definition of the
\csi{emph} command and other things that just \emph{happened} to be in cursive
will be unaffected.
\begin{example}
  \RenewCommandCopy{\emph}{\strong}%!hide
Are you gonna eat \emph{that}?
\end{example}

\subsection{Your First Environment}

If you have played with the \csi{emph} command a bit, you may have noticed that
trying to write several paragraphs inside it results in an error. This is the
case for most \LaTeX{} commands. The reason is that putting a lot of
text inside their parameters could result in poor performance and excessive
memory consumption. In order to overcome that, \LaTeX{} uses a concept of
\emph{environments}.

Environments are started using the \csi{begin} command and ended using the
\csi{end} command. You have already seen one environment --- the
\cargv{document} environment that holds the body of the document. This one is
present exactly once in every \LaTeX{} document. To explore the concept a bit,
let's introduce another one that is not very useful in practice, but is easy to
understand: the \ei{em} environment, short for
\enquote{emphasise}.\footnote{It is not useful because there is very little
  reason to emphasise more than one paragraph. In order to make the emphasis
  effective, it should be used with restraint. After all, if everything is
  emphasised then nothing is.}
\begin{example}
\begin{em}
  This paragraph is emphasised.

  This one is \emph{too}.
\end{em}
\end{example}

\subsection{Summary}

Using logical markup, we can embed the logical structure of our document inside
the document itself. Instead of saying \enquote{write this in green and write
  this word in cursive}, we say \enquote{this text happens in dimension A and
  this word is emphasised}. The style of all text in \enquote*{dimension A},
or of emphasised words, can be decided later and easily changed.

If you started writing your hypothetical novel using \LaTeX{} instead of a
WYSIWYG editor, and used custom environments for typesetting events in different
dimensions, and \csi{emph} for emphasis, then changing it to a black and white
version would come down to simply changing the definition of the custom
environments.

Before learning how to define your own commands and environments, this book
will introduce you to many of the standard ones that are provided either by
\LaTeX{} itself or third-party packages.

\section{Packages}\index{package} While writing your document, you will
probably find that there are some areas where basic \LaTeX{} cannot solve your
problem. If you want to include \wi{graphics}, \wi{coloured text} or source
code from a file in your document, then you need to enhance the capabilities of
%$ 'include' and 'in' or 'within', or 'embed' and 'into'
\LaTeX. Such enhancements are called packages. Packages are activated with the
\begin{lscommand}
  \csi{usepackage}[options: o, package: m]
\end{lscommand}
command, where \carg{package} is the name of the package and
\carg{options} is a list of keywords that trigger special features in
the package. The \csi{usepackage} command goes into the preamble of the
document. See \autoref{sec:structure} for details.

This book will describe some packages that the authors thought were especially
useful and should be installed along with your \LaTeX{} distribution. See
\autoref{packages} for some examples. The versions installed on your system may
be different than
the ones described in this book, which in turn may lead to differences in the
produced output. Along with each package's description we will also point
to its entry in our bibliography. In the bibliography entry, you will find
information about the package version that was used when writing this booklet.
You can check the versions of all packages used in a document by looking
at the \eei{.log} file that is produced when compiling it. If the
package versions aren't very different (usually the first number is the most
important) then you should be fine.

Modern \LaTeX{} distributions come with many packages
preinstalled. If you are working on a \Unix{} system, try using the command
\texttt{texdoc} for accessing package documentation. Alternatively you can
search for the package on \url{https://www.ctan.org/}, and its documentation
should be present under the \enquote*{Documentation} field.

\begin{table}[htp]
  \centering
  \caption{Examples of \LaTeX{} packages.}\label{packages}
  \begin{tabular}{@{}lp{9cm}@{}}
    \toprule
    Package            & Description                                          \\
    \midrule
    \pai*{amsmath}     & Provides additional commands for typesetting
    mathematical symbols, and environments for aligning equations. Described
    in \autoref{chap:math}.                                                   \\

    \pai*{polyglossia} & Makes it easy to write \LaTeX{} documents in
    languages other than English, or even in multiple languages.
    Described in \autoref{sec:polyglossia}.                                   \\

    \pai*{booktabs}    & Provides commands for producing beautifully
    formatted tables for your document. Described in
    \autoref{sec:tables}.                                                     \\

    \pai*{biblatex}    & Provides commands for automatically specifying and
    producing a bibliography for your document. Described in
    \autoref{chap:bibliography}.                                              \\

    \pai*{makeidx}     & Provides commands for producing indexes. Described
    in \autoref{sec:indexing}.                                                \\

    \pai*{fancyhdr}    & Let's you easily customise page headers and footers.
    Described in \autoref{sec:fancy}.                                         \\

    \pai*{beamer}      & Provides a document class that changes output to
    produce presentations and provides command to typeset slides.
    Described in \autoref{sec:beamer}.                                        \\
    \bottomrule
  \end{tabular}
\end{table}

\section{The Structure of Text and Language}
\secby{Hanspeter Schmid}{hanspi@schmid-werren.ch}
The main point of writing a text, is to convey ideas, information, or
knowledge to the reader.  The reader will understand the text better
if these ideas are well-structured, and will see and feel this
structure much better if the typographical form reflects the logical
and semantic structure of the content.

As we have seen, \LaTeX{} is different from other typesetting systems in that
you just have to tell it the logical and semantic structure of a text.  It then
derives the typographical form of the text according to the \enquote{rules}
given in the document class file and in various style files.

The most important text unit in \LaTeX{} (and in typography) is the
\wi{paragraph}.  We call it \enquote{text unit} because a paragraph is the
typographical form that should reflect one coherent thought, or one idea.
Therefore, if a new thought begins, a new paragraph should begin, and if not,
only line breaks should be used.  If in doubt about paragraph breaks, think
about your text as a conveyor of ideas and thoughts.  If you have a paragraph
break, but the old thought continues, it should be removed.  If some totally
new line of thought occurs in the same paragraph, then it should be broken.

Most people completely underestimate the importance of well-placed paragraph
breaks. Many people do not even know what the meaning of a paragraph break is,
or, especially in \LaTeX, introduce paragraph breaks without knowing it.  The
latter mistake is especially easy to make if equations are used in the text.
You have already learned in \autoref{sec:spaces} that paragraph breaks are
introduced by leaving an empty line in the source code. Look at the following
examples, and figure out why sometimes empty lines (paragraph breaks) are used
before and after the equation, and sometimes not. (Ignore the contents of the
equations themselves as they are not important.)

\begin{example}[standalone, paperwidth=5cm, paperheight=4cm]
\begin{document}%!hide
% Example 1
\ldots when Einstein introduced
his formula
\begin{equation}
  e = m \cdot c^2 \; ,
\end{equation}
which is at the same time the
most widely known and the least
well understood physical formula.
\end{document}%!hide
\end{example}
\begin{example}[standalone, paperwidth=5cm, paperheight=4cm]
\begin{document}%!hide
% Example 2
\ldots from which follows
Kirchhoff's current law:
\begin{equation}
  \sum_{k=1}^{n} I_k = 0 \; .
\end{equation}

Kirchhoff's voltage law can
be derived \ldots
\end{document}%!hide
\end{example}
\begin{example}[standalone, paperwidth=5cm, paperheight=4cm]
\begin{document}%!hide
% Example 3
\ldots which has several
advantages.

\begin{equation}
  I_D = I_F - I_R
\end{equation}
is the core of a very different
transistor model. \ldots
\end{document}%!hide
\end{example}

The next smaller text unit is the sentence.  In English texts, there is
a larger space after a period that ends a sentence than after one
that ends an abbreviation.  \LaTeX{} tries to figure out which one
you wanted to have.  If \LaTeX{} gets it wrong, you must tell it what
you want.  This is explained later in the next chapter.

The structuring of text even extends to parts of sentences.  Most
languages have very complicated punctuation rules, but in many
languages (including German and English), you will get almost every
comma right if you remember what it represents: a short stop in the
flow of language.  If you are not sure about where to put a comma,
read the sentence aloud and take a short breath at every comma.  If
this feels awkward at some place, delete that comma; if you feel the
urge to breathe (or make a short stop) at some other place, insert a
comma.

Finally, the paragraphs of a text should also be structured logically at a
higher level, by putting them into chapters, sections, subsections, and so on.

\section{Files You Might Encounter}

When you work with \LaTeX{}, you will soon find yourself in a maze of
files with various \wi{extension}s and probably no clue. The following
list explains the various \wi{file types} you might encounter when
working with \TeX{}. Please note that this table does not claim to be
a complete list of extensions, but if you find one missing that you
think is important, please drop me a line.

\begin{description}

  \item[\eei{.tex}] \LaTeX{} or \TeX{} input file. Can be compiled with
    \texttt{latex}.
  \item[\eei{.sty}] \LaTeX{} Macro package. Load this
    into your \LaTeX{} document using the \csi{usepackage} command.
  \item[\eei{.dtx}] Documented \TeX{}. This is the main distribution
    format for \LaTeX{} style files. If you process a \eei{.dtx} file you get
    documented macro code of the \LaTeX{} package contained in the \eei{.dtx}
    file.
  \item[\eei{.ins}] The installer for the files contained in the
    matching \eei{.dtx} file. If you download a \LaTeX{} package from the net,
    you will normally get a \eei{.dtx} and a \eei{.ins} file. Run \LaTeX{} on the
    \eei{.ins} file to unpack the \eei{.dtx} file.
  \item[\eei{.cls}] Class files define what your document looks
    like. They are selected with the \csi{documentclass} command.
  \item[\eei{.fd}] Font description file, telling  \LaTeX{} about new fonts.
\end{description}

The following files are generated when you run \LaTeX{} on your input
file:
\begin{description}
  \item[\eei{.log}] Gives a detailed account of what happened during the
    last compiler run.
  \item[\eei{.toc}] Stores all your section headers. It gets read in for the
    next compiler run and is used to produce the table of contents.
  \item[\eei{.lof}] This is like \eei{.toc}, but for the list of figures.
  \item[\eei{.lot}] Same again, for the list of tables.
  \item[\eei{.aux}] Another file that transports information from one
    compiler run to the next. Among other things, the \eei{.aux} file is used
    to store information associated with cross-references.
  \item[\eei{.idx}] If your document contains an index, \LaTeX{} stores all
    the words that go into the index in this file. Process this file with
    \texttt{makeindex}. Refer to \autoref{sec:indexing} on
    \autopageref{sec:indexing} for more information on indexing.
  \item[\eei{.ind}] The processed \eei{.idx} file, ready for inclusion into your
    document on the next compile cycle.
  \item[\eei{.ilg}] Log file, telling what \texttt{makeindex} did.
\end{description}
