% The Not So Short Introduction to LaTeX
%
% Copyright (C) 1995--2022 Tobias Oetiker, Marcin Serwin, Hubert Partl,
% Irene Hyna, Elisabeth Schlegl and Contributors.
%
% This document is free software: you can redistribute it and/or modify it
% under the terms of the GNU General Public License as published by the Free
% Software Foundation, either version 3 of the License, or (at your option) any
% later version.
%
% This document is distributed in the hope that it will be useful, but WITHOUT
% ANY WARRANTY; without even the implied warranty of MERCHANTABILITY or FITNESS
% FOR A PARTICULAR PURPOSE.  See the GNU General Public License for more
% details.
%
% You should have received a copy of the GNU General Public License along with
% this document.  If not, see <https://www.gnu.org/licenses/>.

%chktex-file 15
\RequirePackage[colorlinks]{hyperref}
\hypersetup{
  pdfinfo = {
      Title = The Not So Short Introduction to LaTeX,
      Author = {Tobias Oetiker, Marcin Serwin,
          Hubert Partl, Irene Hyna and Elisabeth Schlegl}
    }
}

\usepackage{lshort}
\usepackage{makeidx,shortvrb,latexsym}
\usepackage{biblatex}
\addbibresource{lshort.bib}
\hyphenation{pre-pend-ing hold-er}
%
% This document is ``public domain''. It may be printed and
% distributed free of charge in its original form (including the
% list of authors). If it is changed or if parts of it are used
% within another document, then the author list must include
% all the original authors AND that author (those authors) who
% has (have) made the changes.
%
% Original Copyright H.Partl, E.Schlegl, and I.Hyna (1987).
% English Version Copyright by Tobias Oetiker (1994,1995),
%
% ---------------------------------------------------------------------
%
%
% Formats also with\textt{letterpaper} option, but the pagebreaks might not
% fall as nicely.
%
% To produce a A5 booklet, use a tool like  pstops or dvitodvi
% to  past them together in the right order. Most dvi printer drivers
% can shrink the resulting output to fit on a A4 sheet.
%
\makeindex
\typeout{Copyright T.Oetiker, M.Serwin, H.Partl, E.Schlegl, I.Hyna}

\DeclareMathOperator{\argh}{argh}
\DeclareMathOperator*{\nut}{Nut}

\theoremstyle{definition} \newtheorem{law}{Law}
\theoremstyle{plain}      \newtheorem{jury}[law]{Jury}
\theoremstyle{remark}     \newtheorem*{marg}{Margaret}
\newtheorem{mur}{Murphy}[section]
