% The Not So Short Introduction to LaTeX
%
% Copyright (C) 1995--2022 Tobias Oetiker, Marcin Serwin, Hubert Partl,
% Irene Hyna, Elisabeth Schlegl and Contributors.
%
% This document is free software: you can redistribute it and/or modify it
% under the terms of the GNU General Public License as published by the Free
% Software Foundation, either version 3 of the License, or (at your option) any
% later version.
%
% This document is distributed in the hope that it will be useful, but WITHOUT
% ANY WARRANTY; without even the implied warranty of MERCHANTABILITY or FITNESS
% FOR A PARTICULAR PURPOSE.  See the GNU General Public License for more
% details.
%
% You should have received a copy of the GNU General Public License along with
% this document.  If not, see <https://www.gnu.org/licenses/>.

% !TEX root = ./lshort.tex
% chktex-file 11
\chapter{Things You Shouldn't Use}

\begin{intro}
  \LaTeX{} has been around for well over 30 years now. And it has remained
  largely backward compatible. This means you can easily steal snippets from
  a friends document written ten years ago and use them in your document today.
  In many instances this work without any problem. But even though \LaTeX{} seems happy
  with the old code, you still should not do it. Have a look!
\end{intro}

\begingroup
\ExplSyntaxOn
\NewDocumentCommand{\instead}{mm}{
  \begin{tabular}{@{}l@{}l@{}}
    \toprule
    Instead~of                         & Use \\
    \midrule
    \begin{minipage}[t]{0.5\textwidth}
      #1
    \end{minipage} &
    \begin{minipage}[t]{0.5\textwidth}
      #2
    \end{minipage}        \\
    \bottomrule
  \end{tabular}
  \medskip
  \par\noindent
  \nopagebreak
  \peek_remove_spaces:n {
    \lshort_gobblepars:
  }
}

\NewDocumentCommand{\chto}{+v+v}{
  \begin{trivlist}
    \item
    % Workaround for https://github.com/gpoore/minted/issues/356
    \RenewCommandCopy\VerbEnv\Verbatim
    \RenewCommandCopy\endVerbEnv\endVerbatim
    \begin{minipage}{0.4\textwidth}
      \mintinline{latex}:#1:
    \end{minipage}
    \hspace{\stretch{1}}
    \(\longrightarrow\)
    \hspace{\stretch{1}}
    \begin{minipage}{0.4\textwidth}
      \mintinline{latex}:#2:
    \end{minipage}
  \end{trivlist}
}
\NewDocumentCommand{\vchto}{+v+v}{
  \begin{center}
    % Workaround for https://github.com/gpoore/minted/issues/356
    \RenewCommandCopy\VerbEnv\Verbatim
    \RenewCommandCopy\endVerbEnv\endVerbatim
    \mintinline{latex}:#1:
    \(\big\downarrow\)\\[.2cm]
    \mintinline{latex}:#2:
  \end{center}
}

\NewCommandCopy{\oldsec}{\section}
\RenewDocumentCommand{\section}{m}{\oldsec{\ldots{}~for~#1}}
\ExplSyntaxOff

\section{Display Math}\label{sec:display_math}
\instead{
\ai{\$\$}
} {
\csi{[}, \csi{]} \\
\ei*{equation*}\\
\ei*{displaymath}\\
(All with the \pai{amsmath} package.)
}

\ai{\$\$} is \hologo{plainTeX} syntax and it cannot be modified. The
\ei{displaymath} and \csi{[} are \hologo{LaTeX} commands that are a bit better
in terms of spacing and features, but most importantly they can be redefined.
With \pai{amsmath}, both of them are redefined to be synonyms for the
\ei{equation*} which produces optimal and consistent spacing.

\chto|$$ 2 + 2 = 4 $$||\[ 2 + 2 = 4 \]| %chktex 45

\section{Inline Math}
\instead{
  \ai{\$}
} {
  \csi{(}, \csi{)}
}

Like \autoref{sec:display_math}, \ai{\$} is also \hologo{plainTeX} syntax
and it cannot be redefined. Currently, there are no visual differences between
\ai{\$} and the new version, but it may become an
issue if you want to modify the command yourself. Also, the \csi{(}, \csi{)} commands
can detect math mode nesting and produce an error if it happens.

\chto|$ 2 + 2 = 4 $||\( 2 + 2 = 4 \)|

\section{Typesetting Math}
\instead{
  \csi{over} \\
  \csi{choose} \\
  \csi{overwithdelims} \\
  \csi{atop} \\
  \csi{atopwithdelims} \\
  \csi{above} \\
  \csi{abovewithdelims}
} {
  \csi{frac} \\
  \csi{binom} (\pai{amsmath}) \\
  \csi{genfrac} (\pai{amsmath})
}

These are \hologo{plainTeX} commands for producing fractions, binomial
coefficients and related structures. Due to their unusual syntax, their use may
lead to ambiguous code. It's better to use the commands described in
\autoref{sec:fractions}.
\begin{chktexignore}
\chto|{a \over b}||\frac{a}{b}|
\chto|{a \choose b}|
|\usepackage{amsmath}
% ...
\binom{a}{b}|
\end{chktexignore}

\section{Defining New Commands}\label{sec:def}
\instead{
  \csi{newcommand} \\
  \csi{renewcommand} \\
  \csi{def}
} {
  \csi{NewDocumentCommand} \\
  \csi{RenewDocumentCommand} \\
  \csi{DeclareDocumentCommand}
}

Both \csi{newcommand} and \csi{renewcommand} are \hologo{LaTeX2e} macros. They
are not as expressive as \texttt{...DocumentCommand} family
described in \autoref{sec:new_commands}. Their syntax only allows a single
optional argument with default value (specified as second optional argument)
followed by few mandatory ones (specified as number in first optional
argument).
\begin{chktexignore}
  \vchto|\newcommand{\foo}[4][bar]{ ... }|%
  |\NewDocumentCommand{\foo}{O{bar}mmmm}{ ... }|
\end{chktexignore}

The \csi{def} command is a \hologo{plainTeX} primitive. It \emph{always} defines
the command, even if it was already defined. This is usually not desirable, but
if it's needed you can use \csi{DeclareDocumentCommand}.
\begin{chktexignore}
  \vchto|\def\foo#1#2#3{ ... }|
  |\NewDocumentCommand{\foo}{mmm}{ ... }|
\end{chktexignore}

\section{Copying Commands}
\instead{
  \csi{let}
} {
  \csi{NewCommandCopy} \\
  \csi{RenewCommandCopy} \\
  \csi{DeclareCommandCopy}
}

Like \autoref{sec:def}, \csi{let} is also a \hologo{plainTeX} primitive that
does not guard against accidental redefinition. Moreover it does not work
correctly with some \hologo{LaTeX} commands. Use \csi{NewCommandCopy} as
described in \autoref{sec:copyingcommands}.
\chto|\let\foo\bar||\NewCommandCopy\foo\bar|

\section{Aligning Equations}
\instead{
  \ei*{eqnarray} \\
  \ei*{eqnarray*}
}{
  \ei*{align}  \\
  \ei*{align*} \\
  (Both in \pai{amsmath}.)
}

\ei{eqnarray} and \ei{eqnarray*} are \hologo{LaTeX} environments that allow
aligning equations. However spacing around the binary operators in these environments is far
from ideal. Therefore, it is recommended to always use the \ei{align} environments
from \pai{amsmath} or \ei{IEEEeqnarray} as described in
\autoref{sec:aligned_equations}.
\begin{chktexignore}
\chto
|\begin{eqnarray}
  f(x) & = &  1 + 2 \\
  g(x) & > & 52
\end{eqnarray}|
|\usepackage{amsmath}
% ...
\begin{align}
  f(x) & = 1 + 2 \\
  g(x) & > 52
\end{align}|
\end{chktexignore}

\section{Changing Fonts}
\instead{
  \csi{bf} \\
  \csi{rm} \\
  \csi{sf} \\
  \csi{tt} \\
  \csi{it} \\
  \csi{sc} \\
  \csi{sl}
}{
  \csi{bfseries} \\
  \csi{rmfamily} \\
  \csi{sffamily} \\
  \csi{ttfamily} \\
  \csi{itshape}  \\
  \csi{scshape}  \\
  \csi{slshape}
}

These are \hologo{plainTeX} commands. Each of them resets the font to normal
before changing it, so for example bold italics cannot be achieved using them.
Use newer commands as described in \autoref{sec:fontsize}.
\chto|{\bf foo}||\textbf{foo}|
\chto|{\bf foo}||{\bfseries foo}|

\section{Changing Text Alignment}
\instead{
  \ei*{flushleft}  \\
  \ei*{flushright} \\
  \ei*{center}
}{
  \ei*{FlushLeft} \\
  \ei*{FlushRight} \\
  \ei*{Center} \\
  (\pai{ragged2e})
}

The default \hologo{LaTeX2e} environments for changing text alignment make it
nearly impossible to hyphenate words inside them. Therefore, it is recommended
to use the \pai{ragged2e} equivalents, as described in
\autoref{sec:ragged}, that make the text less \enquote*{ragged} than it
should be.
\begin{chktexignore}  
\chto|\begin{center}
  text
\end{center}|
|\usepackage{ragged2e}
% ...
\begin{Center}
  text
\end{Center}|
\end{chktexignore}

\section{Typesetting Quotations}
\instead{
  \texttt{``} \\
  \texttt{,,} \\
  \texttt{<<} \\
  \texttt{>>} \\
  \texttt{''}
}{
  \csi{enquote} \\
  \csi{enquote*}
}

The \TeX{} version of entering quotes was to rely on ligatures for the given
quotation mark. This method is not context aware and cannot be customized.
Using \pai{csquotes} package as described in \autoref{sec:csquotes} allows
greater control over the typesetting of the quotations.

\chto|``quote''||\usepackage{csquotes}
% ...
\enquote{quote}|

\section{Printing Verbatim}
\instead{
  \ei{verbatim}
}{
  \ei{verbatim} (from the \pai{verbatim} package)
}

\LaTeX{} comes with a \ei{verbatim} environment built into the core, however there are problems
with it. Firstly, due to the way delimiting is handled you cannot put spaces
\begin{chktexignore}
  between \mintinline{latex}|\end| and \mintinline{latex}|{verbatim}|.
\end{chktexignore}
The second and more important problem is that it reads the whole input at once
which risks overflowing \TeX{}'s memory. The \pai{verbatim} package described
in \autoref{sec:verbatim} solves both of these problems.
\begin{chktexignore}
\chto|\begin{verbatim}
  code
\end{verbatim}||\usepackage{verbatim}
% ...
\begin{verbatim}
  code
\end{verbatim}|
\end{chktexignore}

\section{Adding a Bibliography}
\instead{
  \ei{natbib} and \texttt{bibtex}
}{
  \ei{biblatex} and \texttt{biber}
}

As was already mentioned in the \autoref{chap:bibliography} before
\texttt{biber} there was \texttt{bibtex}. It used the \ei{natbib} package for
typesetting the bibliography itself. Both are no longer maintained and
don't handle UTF-8 input, so they should be avoided. However, the \ei{natbib}
styles (\eei{.bst} files) don't work with \ei{biblatex}, so if you are forced to
use such a bibliograph style, you don't have a choice.
\chto|\usepackage{natbib}||\usepackage{biblatex}|

\section{Creating Relations and Operators}
\instead{
  \csi{stackrel} \\
  \csi{stackbin}
}{
  \csi{overset}
}

The commands are primitives that always define the underlying symbol to be
either a relation or binary operator. The \csi{overset} detects this based on
the symbol that is overset. See \autoref{sec:math_accents} for an example of
usage.
\chto|\stackrel{f}{=}||\overset{f}{=}|
\chto|\stackbin{f}{+}||\overset{f}{+}|

\section{Changing Math Font}
\instead{
  \csi{mathrm}  \\
  \csi{mathit}  \\
  \csi{mathbf}  \\
  \csi{mathcal} \\
  \csi{mathscr} \\
  \csi{mathbb}  \\
  \csi{mathfrak}
}{
  \csi{symrm}   \\
  \csi{symit}   \\
  \csi{symbf}   \\
  \csi{symcal}  \\
  \csi{symscr}  \\
  \csi{symbb}   \\
  \csi{symfrak} \\
  (\pai{unicode-math} package)
}

When \LaTeX{} typesets operators it uses \mintinline{latex}{\math...} commands
to typeset their name, usually \csi{mathrm}. This uses a different font
compared to the symbols used in mathematics.
\begin{example}
\(\mathit{ffi}\) vs.\
  \(\symit{ffi}\)
\end{example}
These commands should not be used for changing the font for symbols. For
historical reasons, some of the \mintinline{latex}{\math...} commands are
actually just synonyms for the symbol versions. For example there is no
\csi{mathscr} version for math operators, so it is just a synonym for
\csi{symscr}. For operators it is better to use the \csi{DeclareMathOperator}
command.

\vchto|\(\mathrm{foo}(\mathcal{F})\)||\DeclareMathOperator{\foo}{foo}
% ...
\(\foo(\symcal{F})\)|

\section{Spacing}
\instead{
  \csi{vskip} \\
  \csi{hskip} \\
  \csi{mskip} \\
  \csi{kern}  \\
  \csi{mkern}
}{
  \csi{vspace} \\
  \csi{hspace} \\
  \csi{mspace} (\pai{amsmath} package)
}

These commands are \TeX{} primitives that use non-standard syntax, which may
lead to weird errors. It is preferable to use the \LaTeX{} equivalents, which
use the standard syntax. The \csi{kern} primitive acts as a vertical space when used between two paragraphs, and as a horizontal space when used inside a paragraph.
\begin{chktexignore}  
\chto|\vskip 1cm||\vspace{1cm}|
\chto|\mskip 1mu||\mspace{1mu}|
\end{chktexignore}

\endgroup
